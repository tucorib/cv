% Style classic en bleu

\documentclass[11pt,a4paper]{moderncv}
\moderncvtheme[purple]{classic}
\usepackage[utf8]{inputenc}
\usepackage[top=1.1cm, bottom=1.1cm, left=2cm, right=2cm]{geometry}

% Largeur de la colonne pour les dates
\setlength{\hintscolumnwidth}{2.5cm}

\firstname{Nicolas}
\familyname{Ribeiro}
\title{Data scientist, Computer scientist\newline \large{Ingénieur confirmé (9
ans d'expérience)}\newline \textcolor{white}{Keywords: NLP, IR}}
\address{52 Rue de la Gravette}{31300 Toulouse}    
\email{nicolas.ribeiro.qc.ca@gmail.com}                      
\mobile{\href{tel:+33618748298}{+33 6 18 74 82 98}} 

% Custom commands
\newcommand*{\cvdoublelistitem}[4][.25em]{%
\cvitem[#1]{#2}{\listitemsymbol\begin{minipage}[t]{\listdoubleitemcolumnwidth}#3\end{minipage}%
  \hfill% fill of \separatorcolumnwidth
  \ifthenelse{\equal{#4}{}}%
    {}%
    {\listitemsymbol\begin{minipage}[t]{\listdoubleitemcolumnwidth}#4\end{minipage}}}}

\begin{document}

\hypersetup{pdfauthor={Nicolas Ribeiro},%
            pdfsubject={Data scientist, Computer scientist},%
            pdftitle={Nicolas Ribeiro - CV},%
            pdfkeywords={NLP, IR, Data scientist, Computer scientist},%
            pdfproducer={LaTeX},%
            pdfcreator={pdfLaTeX}
}

\maketitle

\section{\textbf{Atouts industriels}}
\cvitem{Linguistique}{Etude de problématiques linguistiques}
\cvitem{Informatique}{Conception et développement de solutions informatiques
\textit{backend}}
\cvitem{Mathématiques}{Maîtrise d'outils mathématiques tels que l'analyse
fonctionnelle et les statistiques}

\section{\textbf{Compétences techniques}}
\cvitem{Programmation}{Java, Python, SQL (> 15 langages déjà utilisés)}
\cvitem{NLP}{IR, Topic modeling, POS tagging, expressions régulières}
\cvitem{Big data}{Deep learning, Machine learning}
% \cvitem{Outils}{Git, Basecamp, Eclipse}
% \cvitem{Rédaction}{\LaTeX, suite Office}
\cvitem{Systèmes}{Unix, Windows}

\section{\textbf{Aptitudes personnelles}}
\cvdoublelistitem{Langues}{Anglais: Courant\newline \small{TOEIC 2008:
895/990}}{Espagnol: Occasionnel\newline \small{DCL 2014: B1}}
\cvdoublelistitem{Travail}{Collaboration\newline \small{Travail en équipe: 7
ans}}{Autonomie\newline \small{Télétravail: 3 ans}}

\section{\textbf{Formations disciplinaires}}
\cvitem{2012 -- 2014}{Formation de linguiste en NLP, \textit{Université
Toulouse - Jean Jaurès}, Toulouse\newline \small{Master ECIL, recherche en
traitement automatique du langage}}
\cvitem{2010 --
2012}{Etudes de lettres modernes, \textit{Université de Nantes}, Nantes\newline
\small{Licence de Lettres Modernes, initiation à la linguistique}}
\cvitem{2005 -- 2008}{Formation d'ingénieur en
informatique, \textit{ENSEEIHT}, Toulouse\newline \small{Diplôme d'Ingénieur en
Informatique et Mathématiques Appliquées}}
\cvitem{2003 -- 2005}{Préparation à
l'entrée aux Grandes Ecoles, \textit{Lycée Jean Dautet}, La Rochelle\newline
\small{Classes Préparatoires aux Grandes Ecoles, spécialité mathématiques}}

\section{\textbf{Evolution professionnelle}}
\cvitem{2012 -- 2017}{Ingénieur confirmé (NLP), \textit{Safety Data -
CFH}, Toulouse\newline \small{Conception et développement d'une application web
d'analyse automatique de rapports d'incidents.}}
\cvitem{2010 -- 2012}{Ingénieur auto-entrepreneur (informatique), Nantes\newline
\small{Conception et développement d'applications pour smartphones
Android.\newline (chaine de cinémas, agences immobilières, garages
automobiles)}} \cvitem{2008 -- 2010}{Ingénieur débutant
(informatique), \textit{Atos Origin}, Toulouse\newline \small{Développement
de logiciels de gestion pour le compte d'entreprises externes.\newline
(financement de projets aéronautiques, conception d'assemblage automobile)}}

\section{\textbf{Projets personnels}}
\cvitem{Analyse de publications}{Développement d'un outil d'analyse sémantique
et temporelle de publications.\newline \small{Corpus: articles journalistiques,
posts de réseaux sociaux}}
\cvitem{Générateur de journal}{Développement d'un outil
d'aggrégation de publications.\newline \small{Corpus: articles journalistiques,
posts de réseaux sociaux}}

\end{document}

